\chapter{Proceso de migración}\label{ch:migración}

En este punto, se llevará a cabo un análisis exhaustivo de la situación de partida, con el propósito de obtener un conocimiento detallado de las arquitecturas de los hardware disponibles, la variedad de documentos, aplicaciones y tipos de archivos existentes, junto con otros aspectos relevantes. Este análisis tiene como objetivo evitar ajustes imprevistos durante el proceso de migración y establecer un plan de acción con la debida anticipación.

El propósito primordial de este análisis consiste en identificar los requisitos funcionales que el nuevo sistema operativo debe cumplir. Mediante este enfoque, se busca garantizar que el sistema operativo seleccionado sea capaz de satisfacer las necesidades y demandas específicas del entorno, asegurando así una transición exitosa y efectiva hacia el nuevo entorno informático.
	
	\section{Aspectos importantes}
					
		En el análisis de la situación de partida, se identificaron varios aspectos importantes que requieren una atención detallada para la migración exitosa. Entre ellos se encuentran:
		
		\begin{itemize}
			
			\item\textbf{Documentos y sus formatos:} Se ha recopilado información sobre los diferentes tipos de documentos utilizados en el entorno actual, así como sus formatos específicos. Este conocimiento será fundamental para asegurar que el nuevo sistema operativo pueda manejar y ser compatible con estos formatos, evitando la pérdida de información o la necesidad de conversiones complicadas.
		
			\item\textbf{Archivos de audio/videos y sus formatos:} Se ha evaluado la variedad de archivos de audio y video presentes en el sistema. Es esencial verificar que el nuevo sistema operativo cuente con las herramientas y códecs necesarios para reproducir y trabajar con estos archivos de manera efectiva.
		
			\item\textbf{Aplicaciones y sus interfaces:} Se ha analizado el conjunto de aplicaciones utilizadas actualmente y sus interfaces. Es crucial determinar si existen alternativas de software libre que puedan reemplazar estas aplicaciones y proporcionar una experiencia de usuario similar o mejor.
		
			\item\textbf{Bases de datos y estructura de datos:} Se ha estudiado la estructura y el contenido de las bases de datos utilizadas en el sistema actual. Este análisis permitirá planificar la migración de datos hacia el nuevo entorno y asegurarse de que la nueva plataforma sea compatible con las bases de datos existentes.
		
			\item\textbf{Disponibilidad de datos y aplicaciones:} Se ha verificado la disponibilidad de datos y aplicaciones críticas para el funcionamiento del sistema. Es fundamental asegurar que todos los datos y aplicaciones necesarios estarán disponibles en el nuevo entorno para garantizar la continuidad de las operaciones.
		
			\item\textbf{Hardware disponible y drivers necesarios:} Se ha realizado un inventario del hardware existente en el entorno actual. Es importante identificar si el nuevo sistema operativo es compatible con el hardware existente y si se requerirán controladores (drivers) adicionales para su correcto funcionamiento.
		
		\end{itemize}

		Considerando estos aspectos, se podrá establecer un plan de acción detallado para la migración hacia el nuevo sistema operativo, asegurando que se cubran todas las necesidades y se minimicen los inconvenientes durante el proceso de transición. La atención cuidadosa a estos aspectos clave permitirá lograr una migración exitosa y una adaptación fluida al nuevo entorno basado en software libre.\par
			
	
		\vspace{0.3cm}
		
	\section{Inventario de Software}
		
		El objetivo del inventario de software es elaborar un listado completo de todos los programas, aplicaciones, servicios y configuraciones utilizados en los equipos que requieren migración. Este proceso tiene como propósito registrar de manera detallada todo el software presente en los sistemas, lo cual facilitará la planificación y ejecución de la migración de manera ordenada y estructurada.\par
		
		El inventario de software contendrá información esencial, como el nombre de cada programa, su versión, el tipo de aplicación o servicio que representa y las configuraciones específicas realizadas en cada caso. A través de esta recopilación exhaustiva de datos, se obtendrá una visión clara y completa de los componentes del software existente, lo que resultará fundamental para llevar a cabo una migración efectiva y sin contratiempos.\par
		
		Al contar con este inventario detallado, el equipo encargado de la migración podrá tomar decisiones informadas y estratégicas en cada etapa del proceso, asegurando la compatibilidad y disponibilidad de todas las aplicaciones y servicios críticos en el nuevo entorno. Asimismo, esta herramienta proporcionará una base sólida para realizar pruebas y verificar la funcionalidad de cada componente después de la migración, minimizando los riesgos y permitiendo una adaptación fluida al nuevo sistema basado en software libre.\par
		
		\vspace{0.3cm}
		
		\subsection{Software Invetariado}
			
			\begin{itemize}
				
				\item Windows 10 64 bits
				\item Navegador Chrome
				\item Microsoft Outlook,
				\item Adobe After Effect
				\item Adobe Acrobat
				\item Adobe Photoshop
				\item atube gatcher
				\item uTorrent
				\item Microsoft Office 2016
				\item Panda Cloud Antivirus
				\item Windows Media
				\item WinRAR
				\item CCleaner
			
			\end{itemize}
		
			\vspace{0.3cm}
		
	\section{Inventario de Hardware}
			
		El objetivo del inventario de hardware es obtener un conocimiento detallado de las características y especificaciones del hardware de los ordenadores que se planean migrar. Este proceso permite determinar si los equipos contarán con soporte nativo por parte de las distribuciones de software libre y también identificar la necesidad de actualizar algún componente o si existen posibles incidencias relacionadas con el soporte de hardware.\par
		
		En el inventario de hardware se registran datos como el modelo y fabricante de cada equipo, la cantidad de memoria RAM, el tipo y capacidad del disco duro, la tarjeta gráfica, la tarjeta de red, entre otros componentes relevantes. Además, se investiga la disponibilidad de los controladores necesarios para el correcto funcionamiento del hardware en las distribuciones de software libre consideradas para la migración.\par
		
		Con este inventario detallado, se anticipan y abordan posibles desafíos relacionados con el hardware durante el proceso de migración, asegurando así una transición exitosa hacia el software libre.\par
		
		
		\subsection{Hardware Inventariado}
		
			\textbf{Desktop}
							
			Cuatro equipos  con procesador Intel:\par\vspace{0.2cm}
				
			\begin{center}
				\begin{tabular}{| r | l |}
					\hline
					Hardware & Modelo \\ \hline
					Mother: & Asrock H55M-LE \\ 
					Procesador: & Intel Core i3 3.0GHz. \\
					Disco Duro & 500GB. \\
					Memoria RAM & DDR3 2GB \\
					Arquitectura & 64bits.\\ \hline
				\end{tabular}
			\end{center}
		
			\vspace{0.3cm}
		
			Tres equipos con procesador AMD:\par\vspace{0.2cm}
					
			\begin{center}			
				\begin{tabular}{| r | l |}
					\hline
					Hardware & Modelo \\ \hline
					Mother: & MSI \\ 
					Procesador: & Amd athlon2 x2 3.00GHz. \\
					Disco Duro & 500GB. \\
					Memoria RAM & DDR3 2GB \\
					Arquitectura & 64bits.\\ \hline
				\end{tabular}
			\end{center}
			
			\vspace{0.3cm}
		
			\textbf{Servidor}	

		
			\begin{center}			
				\begin{tabular}{| r | l |}
					\hline
					Hardware & Modelo \\ \hline
					Mother: & FM2 \\ 
					Procesador: & Amd 3.5ghz. \\
					Disco Duro & 500GB. \\
					Memoria RAM & DDR3 8GB \\
					Arquitectura & 64bits.\\ \hline
				\end{tabular}
			\end{center} 
					
		\vspace{0.3cm}
		
	
	\section{Sistema Operativo}
		
		\subsection{Distribución GNU/Linux}
			
		Las distribuciones GNU/Linux representan sistemas operativos basados en el núcleo Linux, que contienen una selección de paquetes de software. Estas distribuciones están mayormente compuestas por software libre, aunque en algunos casos también pueden incluir software propietario.\par
		
		Adicionalmente al núcleo Linux, las distribuciones GNU/Linux típicamente integran las bibliotecas y herramientas del proyecto GNU, así como el sistema de ventanas X Window System. En función de su enfoque y el público objetivo al que van dirigidas, estas distribuciones pueden incluir una diversidad de aplicaciones y herramientas, como procesadores de texto, hojas de cálculo, reproductores multimedia y utilidades administrativas.\par
		
		Es relevante enfatizar que cuando una distribución GNU/Linux incorpora paquetes del proyecto GNU, se le conoce como distribución GNU/Linux, en reconocimiento a la contribución del software libre desarrollado por el proyecto GNU dentro de la distribución.\par	
					

		\subsection{Debian}\label{sub:debian}
			

			Se seleccionó Debian como sistema operativo para el servidor de virtualización debido a sus características y ventajas. Debian es una distribución mantenida por la comunidad que ofrece una excelente estabilidad y facilita las actualizaciones de paquetes y del propio sistema de manera sencilla.

			Debian es conocido por su amplio soporte en diversas arquitecturas y dispositivos, además de ofrecer un soporte a largo plazo (LTS).

			Entre las características destacadas de Debian se encuentran las siguientes:
		
		\begin{itemize}
		
		
			\item Es una distribución compuesta únicamente por software libre, y se compromete a mantener esa condición en todas sus versiones.
			
			\item Ofrece una gran estabilidad y seguridad, al estar basado en el núcleo Linux. Proporciona una configuración 
			predeterminada para cada paquete y ofrece actualizaciones de seguridad de forma regular durante todo su ciclo de vida.
			
			\item Cuenta con un amplio soporte de hardware, siendo compatible con la mayoría de los dispositivos compatibles con el núcleo Linux. En caso de ser necesario, Debian cuenta con controladores adicionales que facilitan la compatibilidad con otros dispositivos.
		
		
		\end{itemize}

			
	\section{Instalación del sistema operativo}
			
	 		Antes de iniciar la instalación del sistema operativo en los ordenadores, se llevará a cabo la tarea de respaldar toda la información de los alumnos en un disco externo. Esta acción es considerada esencial para preservar los datos de manera segura y evitar pérdidas.
	 		
	 		Se presentan varias opciones para realizar el respaldo, pero siempre se recomienda utilizar un medio físico de almacenamiento adicional. Esto puede incluir discos duros externos, unidades USB, discos ópticos u otros dispositivos de respaldo adecuados. El objetivo es garantizar que los datos estén protegidos y disponibles para su posterior recuperación en caso de ser necesario.\par 
			
	    %	\begin{tcolorbox}[enhanced,attach boxed title to top center={yshift=-3mm,yshifttext=-1mm},
		%		colback=blue!5!white,colframe=blue!75!black,colbacktitle=red!80!black,title= Debian,fonttitle=\bfseries, boxed title style={size=small,colframe=red!50!black} ]
		
		%		\centering
		
		%		\href{https://cdimage.debian.org/debian-cd/current/amd64/bt-dvd/}{\color{blue}{}Descargar Debian	}
			
		%	\end{tcolorbox}
				
			\subsection{Requisitos}
				
				Los requisitos mínimos necesarios para la instalación de \textbf{Debían Cinnamon} son los siguientes:
				
				\begin{itemize}
					
					\item 1 GB de RAM (se recomiendan 2 GB para un uso cómodo).
					\item 15 GB de espacio en disco (se recomiendan 20 GB).
					\item Resolución de 1024×768.
					\item Procesador 2Ghz, doble núcleo.
					
				\end{itemize}
			
			\subsection{Proceso de Instalación}
			
			El proceso de instalación de Debian con el entorno de escritorio Cinnamon se lleva a cabo siguiendo los siguientes pasos:
			
			
			\begin{itemize}
				
				\item \textbf{Preparación del medio de instalación:} En primer lugar, se obtiene la imagen ISO de Debian con el entorno Cinnamon desde el sitio web oficial de Debian. Luego, se graba la imagen en un DVD o se crea un dispositivo USB de arranque.
					
				\item \textbf{Arranque del sistema desde el medio de instalación:} Una vez que el medio de instalación está listo, se inicia el ordenador desde este medio. En la mayoría de los casos, esto implica configurar la secuencia de arranque en la BIOS o UEFI para que el equipo inicie desde el DVD o USB.
					
				\item \textbf{Selección del idioma y configuración:} Al iniciar el sistema desde el medio de instalación, se presentará un menú de bienvenida donde se puede seleccionar el idioma preferido. A continuación, se eligen las configuraciones regionales adecuadas, como la zona horaria y el teclado.
					
				\item \textbf{Configuración de la red:} Si se necesita acceso a internet durante la instalación (por ejemplo, para descargar paquetes adicionales), se configura la conexión a la red.
					
				\item {Particionado del disco:} Luego, se procede a la configuración del particionado del disco duro. Aquí, se puede elegir entre diferentes opciones, como usar todo el disco para Debian o realizar un particionado personalizado. Es importante tener cuidado, ya que esta etapa puede borrar datos existentes si no se realiza con precaución.
					
				\item Selección de paquetes de software: En este paso, se elige el entorno de escritorio Cinnamon junto con otros paquetes adicionales que se deseen instalar. Además, se puede seleccionar el tipo de instalación, como una instalación estándar o personalizada.
					
				\item \textbf{Configuración del gestor de arranque:} Se selecciona el gestor de arranque que se utilizará para iniciar Debian. Por lo general, GRUB (Grand Unified Bootloader) es el gestor de arranque predeterminado y adecuado para la mayoría de las configuraciones.
					
				\item \textbf{Creación del usuario y contraseña:} Se crea una cuenta de usuario con un nombre de usuario y una contraseña para acceder al sistema.
					
				\item \textbf{Finalización de la instalación:} Una vez que se han realizado todas las selecciones y configuraciones, se confirma la instalación y el proceso de instalación de Debian con Cinnamon comienza. El sistema copiará todos los archivos necesarios y configurará los paquetes seleccionados.
					
				\item \textbf{Reinicio del sistema:} Al finalizar la instalación, se reinicia el sistema y se cargará Debian con el entorno Cinnamon. A partir de ese momento, el usuario podrá acceder a su cuenta y comenzar a utilizar Debian con Cinnamon como entorno de escritorio.
			
		\end{itemize}
	
	\section{Repositorios}
	
		\subsection{¿Que es un repositorio?}
		
			Un repositorio es un lugar centralizado y organizado donde se almacenan y mantienen los archivos y paquetes de software de una aplicación, programa o sistema operativo. En el contexto de sistemas operativos basados en Linux, como Debian, Ubuntu, CentOS, entre otros, los repositorios son fundamentales para la instalación y actualización de software de manera sencilla y segura.
			
			En un repositorio, se encuentran diversos tipos de software, como aplicaciones, bibliotecas, controladores y actualizaciones del sistema. Cada paquete de software está acompañado por metadatos que proporcionan información sobre su versión, autor, dependencias y otras características relevantes.
			
			Los repositorios pueden ser oficiales, proporcionados por los desarrolladores del sistema operativo, o de terceros, creados y mantenidos por la comunidad o empresas externas. La utilización de repositorios oficiales y de confianza es esencial para garantizar la seguridad y estabilidad del sistema, ya que se asegura de que el software provenga de fuentes confiables y se someta a controles de calidad antes de su inclusión.\par
			
		\subsection{Configuración de repositorios Debian}
		
		La lista que requiere edición se encuentra alojada en \textit{/etc/apt/sources.list}. Para obtener información acerca de los repositorios disponibles proporcionados por \href{https://www.debian.org}{\color{blue}{}Debian}, se puede consultar una \href{https://wiki.debian.org/SourcesList/}{\color{blue}{}Wiki Debian} que cuenta con información detallada sobre cada uno de los repositorios.\par
		
	
			
	\section{Instalación de Programas}
		
		La mayoría de los programas necesarios para los estudiante se instalan de forma automática, mientras que otros deberán ser instalados manualmente a través de la terminal.\par

		A continuación, se detallan los programas que deben ser instalados en los equipos:
		
		\begin{itemize}
			\item \textbf{Brave:} Brave es un navegador totalmente gratuito y de código abierto para ordenadores o teléfonos móviles, que destaca por su enfoque en la privacidad y velocidad.

			\item\textbf{Chromium:} Chromium es una versión de código abierto de Google Chrome, pero sin los códecs exclusivos y otros elementos que Google utiliza para diferenciar Chrome de otros navegadores.

			\item \textbf{Cliente de correo Thunderbird:} Thunderbird es el cliente de correo electrónico desarrollado por la Fundación Mozilla, responsable también del navegador Firefox. Thunderbird se ha diseñado para cubrir las necesidades de aquellos que buscan un gestor de correo electrónico ligero y gratuito.
			
			\item\textbf{Evince:} Evince es un visor de documentos para el entorno de escritorio GNOME que permite visualizar archivos en formato PDF y PostScript.
			
			\item\textbf{Gimp:} GIMP (GNU Image Manipulation Program) es un programa de edición de imágenes de software libre, que forma parte del proyecto GNU y se distribuye bajo licencia pública y GNU Lesser General Public License.
			
			\item\textbf{Inkscape:} Inkscape es un editor de gráficos vectoriales que permite diseñar imágenes de calidad, tanto básicas como complejas. Ofrece herramientas para crear y editar diagramas, líneas, gráficos, logotipos, cómics, folletos, entre otros elementos.
			
			\item\textbf{HandBrake:} HandBrake es un programa de software libre que permite editar archivos de audio y video.
			
			\item\textbf{qBittorrent:} qBittorrent es un cliente P2P de software libre que se utiliza para la transferencia de archivos grandes.
			
			\item\textbf{FileZilla:} FileZilla es un programa de software libre que funciona a nivel cliente/servidor, permitiendo conectarse a un servidor para consultar, adquirir y manipular contenido del mismo.
			
			\item\textbf{VLC:} VLC es un reproductor y framework de video y música de software libre, compatible con una amplia gama de formatos multimedia. Es capaz de reproducir la mayoría de los códecs sin necesidad de descargar paquetes adicionales.
			
			\item\textbf{BleachBit:} BleachBit es una herramienta de software libre que se encarga de eliminar elementos como caché, cookies, archivos temporales, historiales, registros de chats, miniaturas y accesos directos inválidos.
			
			\item\textbf{TeXstudio:} TeXstudio es un editor de \LaTeX{} multiplataforma de software libre, que ofrece funciones como marcadores, autocompletado de comandos, coloreado de sintaxis, soporte de arrastrar imágenes y asistente para la creación de tablas y fórmulas, entre otras características destacadas.
			
		
		\end{itemize}
		
		Para instalar el navegador Brave es necesario que agregue el repositorio donde se encuentra el navegador
		
		\vspace{0.3cm}
			
		\begin{lstlisting}[language=Bash, caption=Instalación de programas]	
	
		#herramientas necesarias para agregar el repositorio.
		apt install apt-transport-https curl
		
		#repositorio de Brave
		curl -fsSLo /usr/share/keyrings/brave-browser-archive-keyring.gpg https://brave-browser-apt-release.s3.brave.com/brave-browser-archive-keyring.gpg
	
		echo "deb [signed-by=/usr/share/keyrings/brave-browser-archive-keyring.gpg arch=amd64] 		https://brave-browser-apt-release.s3.brave.com/ stable main"| tee /etc/apt/sources.list.d/brave-browser-release.list
		
		#actualizar la lista de paquetes disponibles.
		apt update
		
		#instalar pogramas.
		apt install brave-browser texstudio bleatbit vlc filezilla qbittorrent handbrake inkscape gimp evince thunderbird chromium
		
			
		\end{lstlisting}