\chapter{Contexto}\label{cap:contexto}

    El Programa “Educación Universitaria en Prisiones”.
    El cual consiste en alinea con aquellos intentos de transformar la herramienta educativa
    en un vehículo no ya de “corrección”, ni de “moralización”, sino de resistencia frente
    a la degradación cotidiana que el encierro supone. Se trata siempre de 
    intentar construir espacios de libertad, gobernados por una lógica sustancialmente 
    distinta de aquella que rige el penal.\par
    El Programa comenzó a funcionar en el año 2004, a partir de la firma de un convenio
    entre la Universidad Nacional del Litoral y el entonces Ministerio de Gobierno, 
    Justicia y Culto. En función de este convenio, se disponía la instalación de 
    aulas virtuales en las Unidades Penitenciarias N º I de la Ciudad de Coronda y 
    N º II “Las Flores” de la ciudad de Santa Fe.\par
    Estas aulas se integrarían a la Red de Campus Virtuales a través de los cuales 
    opera el Centro Multimedial de Educación a Distancia de la UNL.\par

    \vspace{0,5cm}

    \begin{center}

        Propuesta y actividades desarrolladas en las aulas virtuales
    %\section{Propuesta y actividades desarrolladas en las aulas virtuales }

    \end{center}

    \vspace{0.2cm}


    La oferta educativa del Programa está compuesta por carreras de pre-grado, denominadas
    Tecnicaturas, que brindan formación técnica vinculada con demandas del mercado laboral,
    tienen una duración que oscila entre 5 y 6 cuatrimestres y otorgan título universitario de
    validez nacional.

