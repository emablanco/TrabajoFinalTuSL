\chapter{Estado del arte}\label{cap:estado}

	Luego de realizar una investigación sobre trabajos que plantearan algo similar a lo propuesto, tome como referencia central la \href{https://1library.co/document/4yr1o87q-universidad-tecnica-de-manabi-facultad-de-ciencias-informaticas.html}{\color{blue}Tesis} de la Universidad Técnica de Manabí “Instalación y Configuración de Equipos Informáticos bajo software Libre”\par
	
	El objetivo de este documento es ofrecer a la biblioteca de su facultad una alternativa al software privativo y realizar una migración exitosa de los entornos de escritorio a software libre. Este documento es una guía de buenas prácticas. El mismo ofrece una visión acerca de los pasos, procesos y elementos necesarios para migrar los equipos de la biblioteca a entornos que solamente utilicen software libre. \par
	
	Este trabajo cubre por una parte lo planteado con la desventaja que nunca fue actualizado, es por ello, que decidí realizar un trabajo nuevo, detallando la migración de una red de computadoras y que aplicativos de software utilizar para que sirva como referencia para futuras migraciones de establecimientos con las mismas características. \par
	
	El Software que se utilizara en esta migración fue seleccionado por poseer licencias de Software Libre y sus respectivas ventajas practicas con sus contra partes privativas.\par