\chapter{Estado del arte}\label{cap:estado}

	Tras realizar una exhaustiva investigación sobre trabajos relacionados con la migración a software libre, se tomó como referencia central \href{https://1library.co/document/4yr1o87q-universidad-tecnica-de-manabi-facultad-de-ciencias-informaticas.html}{\color{blue}Tesis} de "Migración a Software Libre" llevado a cabo por la Universidad Técnica de Manabí, específicamente el trabajo titulado "Instalación y Configuración de Equipos Informáticos bajo software Libre".\par 

El objetivo principal de este documento es ofrecer a la biblioteca de su facultad una alternativa al software propietario y lograr una migración exitosa de los entornos de escritorio hacia el software libre. Se trata de una guía de buenas prácticas que proporciona una visión clara de los pasos, procesos y elementos necesarios para llevar a cabo la migración de los equipos de la biblioteca, utilizando exclusivamente software libre.\par 

Si bien este trabajo aborda parte de los aspectos planteados, una desventaja significativa es que no ha sido actualizado. Por lo tanto, se ha decidido realizar un nuevo trabajo que detalla la migración de una red de computadoras y los aplicativos de software que se deben utilizar. El objetivo es que este nuevo trabajo sirva como referencia para futuras migraciones en establecimientos con características similares.\par 

La selección del software para esta migración se ha basado en el uso de licencias de Software Libre y las ventajas prácticas que ofrece en comparación con sus contrapartes propietarias. Se han tenido en cuenta tanto la disponibilidad de las licencias como las ventajas específicas que brinda cada aplicación de software libre seleccionada, en comparación con sus equivalentes privativos.\par 