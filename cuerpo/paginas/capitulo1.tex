\chapter{Introducción}\label{ch:ntroducción}


	Este trabajo es el resultado de un largo periodo formativo, el cual 
	concluyó en Diciembre del 2020. Lo desarrollado en esta página pretende 
	cumplir con los requisitos del Trabajo Final de la Tecnicatura Universitaria 
	en Software Libre.\par
	
	El trabajo desarrollado tuvo lugar en la sala de informática perteneciente 
	a la UNL, ubicada en el instituto de detención penal Nº 2 “Las Flores”. 
	El propósito de esta sala informática es posibilitar el acceso a personas 
	privadas de su libertad a distintas trayectorias educativas/culturales/laborales, 
	mediante el estudio a distancia a través de internet.\par
	
	El siguiente texto describe la migración de Software Propietario a Software Libre 
	realizada en el Aula Virtual de la UNL, en el penal de las flores.\par
	
	Los objetivos propuestos y a concretar fueron los siguientes:\par
	
	\begin{itemize}
	
	    \item Migrar los equipos informáticos de Software Privativos 
	    a Software Libre en todas sus dimensiones.
	      
	    \item Lograr el cambio de una filosofía a otra en la selección y organización 
	    de las tecnologías utilizadas en el aula.
	      
	    \item Posibilitar que los estudiantes modifiquen la visión que tienen 
	    respecto al Software Libre.
	      
	\end{itemize}
	        
	     \vspace{0,5cm}
	
	     Problemas resueltos:
	
	\begin{itemize}
	            
	
	    \item Bajo rendimiento del hardware disponible.
	
	    
	    \item Incompatibilidad del software con el hardware disponible.
	
	    
	    \item Rendimiento deficitario de la red informática.
	
	    
	    \item Falta de autonomía tecnológica.
	
	    
	\end{itemize}
	        
	
	\vspace{0,5cm}
	        
	
	Para elaborar este trabajo fue necesario llevar adelante una investigación 
	previa a la migración, la cual consistió en dos etapas:\par
	
	
	\begin{itemize}
	
	
	    \item Investigar sobre otros procesos de migración o proyectos similares que hayan tenido éxito, indagar si podría tomarse como modelo alguno de ellos/continuarlo/mejorarlo.
	
	        
	    \item Diseñar una estrategia que permita alcanzar el objetivo planteado, junto con un cronograma especificando las acciones y tiempo estimado de inicio y finalización.
	        
	        
	\end{itemize}
	
	
	Todo lo realizado fue posible gracias al apoyo de la gran comunidad del Software Libre. Es el deseo de este estudiante activista que todo lo elaborado pueda servir para motivar y ayudar a otras personas en el proceso de migración de cualquier otra institución con similares características.\par
	
	\clearpage
	
	\begin{center}
	
	    \textbf{Movimiento Software Libre}
	    
	\end{center}
	
	
	\vspace{0.2cm}
	
	
	En la década del ‘70’, cuando la computación estaba en sus inicios era común que tanto los desarrolladores de software profesionales, como los	aficionados, publicaran sus trabajos para que otros puedan utilizarlo, corregir errores y mejorarlo. A partir del avance de las industrias tecnologías sobre el software en la década del ‘80’, las practicas 
	colaborativas se vieron afectadas,  y en consecuencia muchos desarrolladores de software decidieron dejar de compartir sus trabajos y solamente dejar que otras personas lo utilicen bajo ciertas condiciones, manifiestas en lo que se denomina “licencias restrictivas”. Estas 
	licencias no permiten que se pueda compartir el programa sin el consentimiento del desarrollador y mucho menos la posibilidad de poder hacerle modificaciones para corregir errores o agregar mejoras.\par
	
	Un caso ejemplar lo constituye el episodio protagonizado por la empresa Microsoft, la cual le envió una carta a un grupo de programadores	aficionados que utilizaban copias no autorizadas de su programa BASIC. En esta carta, Bill Gates, “General Partner”, acusa a esos programadores 
	de que le están robando su programa, argumentando que compartir el software es injusto, ya que su creador no recibe suficiente dinero a cambio. Esta forma de pensar atentaba contra el espíritu de cooperación, solidaridad y reciprocidad que existía en ese entonces en los grupos informáticos.\par
	
	Para contrarrestar esta tendencia a no compartir el código fuente, surgió
	el movimiento Software Libre.\par
	
	
	El Software Libre es un movimiento ético, político y social, que tiene por objetivo defender la libertad de las personas en un mundo donde las computadoras afectan cada vez más nuestra forma de vivir. Se lo considera como un movimiento político y social, dado que no solo implica defender las “cuatros libertades esenciales”, sino que también, el no permitir que el capitalismo tome el poder del conocimiento. Es por ello, que centra su lucha en una mirada política sobre el conocimiento en general y las tecnologías en particular, ya que se cuestiona el concepto de “propiedad privada del conocimiento” y busca promover la libertad de los “usuarios de computadoras”, para contribuir en la lucha por los derechos de los ciudadanos en el entorno digital. ( traficante de sueños)\par
	
	Los artefactos diseñados bajo la filosofía del Software Libre posibilita Satisfacer las necesidades tecnológicas de las comunidades y los individuos,ya que al poder modificarse se puede adaptar a las necesidades existentes.\par
	
	Ademas, al poder ser redistribuido libremente, (sea la versión original o una con modificaciones) se aporta al desarrollo de la sociedad. De esta manera al compartir el programa y las ideas, se logra generar más conocimiento y el involucra miento de las personas en las decisiones sobre el desarrollo tecnológico de sus comunidades.\par
	\vspace{1cm}
