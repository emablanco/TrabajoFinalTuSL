\chapter{Desarrollo}\label{ch:desarrollo}
	
	En el presente trabajo se contempla la migración de la red informática del Aula Virtual, la cual cuenta con 9 estaciones de trabajo. Una de ellas será utilizada como un Servidor Docker donde se ejecutarán los servicios que se ofrecerán en el presente y en el futuro a los estudiantes del espacio. A los demás equipos se les instalará un Sistema Operativo GNU/Linux orientado a usuarios finales, junto con las aplicaciones necesarias para el trabajo diario, como un navegador de internet, cliente de correo electrónico, suite de oficina, cliente FTP y TexStudio.\par

Desde el principio, se ha tenido en cuenta los diferentes formatos de archivos para realizar la migración y evitar así uno de los grandes problemas que se presentan al momento de migrar a software libre. Las aplicaciones de software libre que se utilizarán son capaces de soportar formatos privativos y permiten importarlos a formatos libres. No obstante, al exportar los documentos al formato libre, es posible que se pierdan algunas características.