\chapter{Desarrollo}\label{ch:desarrollo}
	
	
	En el presente trabajo, se contempla la migración de la red informática del Aula Virtual, que consta de nueve estaciones de trabajo. Una de ellas se destinará para funcionar como un servidor de virtualización, en el cual se ejecutarán los servicios que se ofrecerán a los estudiantes del espacio tanto en el presente como en el futuro.\par
	
	Para las demás estaciones de trabajo, se prevé la instalación de un Sistema Operativo GNU/Linux orientado a usuarios finales. En cada uno de estos equipos, se configurarán las aplicaciones necesarias para el trabajo diario, incluyendo un navegador de internet, cliente de correo electrónico, suite de oficina, cliente FTP y TexStudio.\par
	
	Desde el inicio del proyecto, se ha tenido en cuenta la diversidad de formatos de archivos con el objetivo de facilitar la migración y evitar los desafíos comunes asociados con la transición hacia el software libre. Se ha seleccionado cuidadosamente aplicaciones de software libre que tienen la capacidad de soportar formatos privativos y permiten importarlos hacia formatos libres. Sin embargo, es importante señalar que al exportar documentos al formato libre, existe la posibilidad de que se pierdan algunas características específicas.\par
	
	A pesar de esta consideración, la adopción del servidor de virtualización y el uso del software libre en las estaciones de trabajo promoverán una mayor autonomía y flexibilidad dentro del espacio educativo. Esto permitirá a los estudiantes trabajar con herramientas abiertas y transparentes, fomentando la colaboración y la compatibilidad con diferentes sistemas y formatos. Asimismo, la migración hacia el software libre contribuirá a fortalecer la seguridad y estabilidad de la infraestructura informática del Aula Virtual, beneficiando a toda la comunidad educativa.\par