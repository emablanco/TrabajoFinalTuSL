\chapter{Desarrollo}\label{ch:desarrollo}
	
	El presente trabajo contempla la migración de la red informática del Aula Virtual, la cual cuenta con 9 estaciones de trabajo, de las cuales una sera utilizada como un Servidor Docker donde se correrán los servicios que se ofrecerán en el presente/futuro a los estudiantes del espacio. Al resto de los equipos se les instalara un Sistema Operativo GNU/Linux orientado a usuario final, con las aplicaciones requeridas para el trabajo diario:\par
	
	\begin{itemize}
		
		\item Navegador de Internet.
		
		\item Cliente de Correo Electrónico.
		
		\item Suite de Oficina.
		
		\item Cliente FTP. 
		
		\item TexStudio.
				
	\end{itemize}
	
	Desde comienzo se tuvo en cuenta los diferentes formatos de archivos para realizar la migración y así evitar uno de los grandes problemas que se presentan al momento de realizar las migraciones a software libre.\par
	
	Las aplicaciones de software libre que se usaran soportan formatos privativos y permiten importarlos a formatos libres. No obstante, en los documentos al momentos de exportarlos al formato libre se pueden perder algunas características propias del formato privativo, lo cual lleva a tener cuidado al momento de la conversión de formato.\par
	
