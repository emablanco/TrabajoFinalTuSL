\chapter{Conclusión}\label{ch:Conclusión}

	En este plan de migración he considerando los programas privativos que se utilizan actualmente y sus respectivas alternativas en software libre. Según los resultados obtenidos a partir de la investigación y la búsqueda de alternativas libres ,se puede concluir que la migración a software libre es factible si los alumnos están de acuerdo.\par
	
	Cabe destacar que uno de los factores más importantes en la busqueda de información fue la comunidad de Software Libre. La cual esta conformada en su mayoría por usuarios finales. Estos mismos usuarios me sirvieron de guía y ayuda para enfocarme en la elaboración de una estrategia que me permita afrontar el problema propuesto. Con la ayuda de sus recomendaciones concluí que se requiere una capacitación y formación adecuada para trabajar en el nuevo entorno de trabajo. En mi caso, como persona a cargo de la migración e implementación debo asumir el tiempo asociados al entrenamiento, capacitación, formación y soporte a los alumnos.\par
	
	Como trabajo futuro se propone la ampliación de este proyecto, la generación de una política de seguridad, administración centralizada y una planificación más detallada del proceso de migración.
	
	Como dice Stallman, es importante que en escuelas y universidades se utilice software libre, ya que estas instituciones educativas están decidiendo el futuro de la sociedad. Por lo tanto, no se debe	aceptar que en un espacio perteneciente a la universidad se utilice y enseñe a utilizar software privativo, sabiendo que el deber de una universidad es la creación y difusión del conocimiento. Los alumnos como parte de ella, también deben defender la libertad de las personas para compartir el conocimiento y el software.
	
	\vspace{0.3cm}
	
	\begin{quote}
		
		\begin{flushright}
			
		{\small	«Las escuelas deben enseñar a sus alumnos a ser ciudadanos de una sociedad fuerte, capaz, independiente y libre».\par
		
		The Free Software Foundation.}
		
	\end{flushright}

		
	\end{quote}

