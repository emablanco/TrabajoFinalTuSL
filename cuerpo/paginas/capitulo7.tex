\chapter{Licencias involucradas}\label{ch:licencias}

	\section{Tipos de licencia}\vspace{0.6cm}

			\subsection{Licencias GPL}\label{gpl}
			
				Una de las más utilizadas es la Licencia Pública General de GNU (GNU GPL). El autor conserva los derechos de autor, y permite la redistribución y modificación bajo algunos términos diseñados para asegurarse de que todas las versiones modificadas del software permanecen bajo la misma licencia GNU GPL. Esto hace que sea imposible crear un producto con partes no licenciadas GPL: La licencia GNU GPL posibilita la modificación y redistribución del software, pero únicamente bajo esa misma licencia.\par
	
	
			\subsection{Licencias LGPL}\label{lgpl}
				
				La Licencia Pública General Reducida de GNU, o más conocida por su nombre en inglés GNU Lesser General Public License, es una licencia creada por la (FSF) que garantiza la libertad de compartir y modificar el software cubierto por ella, asegurando que el software es libre para todos sus usuarios. Esta licencia se aplica a cualquier programa o trabajo que contenga una nota puesta por el propietario de los derechos del trabajo estableciendo que su trabajo puede ser distribuido bajo los términos de esta.\par
			
			\subsection{Licencias AGPL}\label{agpl}
			
				La Licencia Pública General de Affero es una licencia copyleft derivada de la Licencia Pública General de GNU diseñada específicamente para asegurar la cooperación con la comunidad en el caso de software que corra en servidores de red. Se engloba dentro de las licencias destinadas a modificar el derecho de autor derivadas de GNU. La Affero GPL es íntegramente una GNU GPL con una cláusula nueva que añade la obligación de distribuir el software si este se ejecuta para ofrecer servicios a través de una red de ordenadores. 
				
				La novedad de AGPL es que, aparte de las cláusulas propias de una GNU GPL, ésta obliga a que se distribuya el software que se destine a dar servicios a través de una red de ordenadores, es decir, si se quiere usar como parte del desarrollo de un nuevo software, este quedaría obligado a su libre distribución.\par
				
			\subsection{Licencias Estilo BSD}\label{bsd}
			
				Llamadas así porque se utilizan en gran cantidad de software distribuido junto a los sistemas operativos BSD. Es una licencia permisiva que casi no impone condiciones sobre lo que un usuario puede hacer con el software. El autor, bajo tales licencias, mantiene la protección de copyright únicamente para la renuncia de garantía y para requerir la adecuada atribución de la autoría en trabajos derivados, pero permite la libre redistribución y modificación, incluso si dichos trabajos tienen propietario. Son muy permisivas, tanto que son fácilmente absorbidas al ser mezcladas con la licencia GNU GPL con quienes son compatibles. También, BSD permite el cobro por la distribución de objetos binarios. Otras opiniones están orientadas a destacar que este tipo de licencia no contribuye al desarrollo de más software libre (normalmente utilizando la siguiente analogía: "una licencia BSD es más libre que una GPL si y solo si se opina también que un país que permita la esclavitud es más libre que otro que no la permite").\par
			
			\subsection{Licencia PSFL}\label{psfl}
			
				La Python Software Foundation License, anteriormente Python License, es una licencia de software libre permisiva, al estilo de la licencia BSD. Cumple con los requisitos OSI para ser declarada licencia de software libre; además, es compatible con la licencia GPL. A diferencia de la licencia GPL, y como la mayoría de licencias tipo BSD, la licencia PSFL no es una licencia copyleft, y permite modificaciones del código fuente, así como la creación de trabajos derivados, sin requerir que ni las modificaciones, ni los trabajos derivados tengan que ser a su vez de código abierto. La licencia PSFL está dentro de las listas de licencias aprobadas tanto por la Free Software Foundation como por la Open Source Initiative.\par
				
			\subsection{Licencias MPL y derivadas}\label{mpl}
			
					Esta licencia es de Software Libre y tiene un gran valor porque fue el instrumento que empleó Netscape Communications Corp. para liberar su Netscape Communicator 4.0 y empezar ese proyecto tan importante para el mundo del Software Libre: Mozilla. Se utilizan en gran cantidad de productos de software libre de uso cotidiano en todo tipo de sistemas operativos. La MPL es Software Libre y promueve eficazmente la colaboración evitando el efecto "viral" de la GPL. No obstante la MPL no es tan excesivamente permisiva como las licencias tipo BSD. Estas licencias son denominadas de copyleft débil. La NPL (luego la MPL) fue la primera licencia nueva después de muchos años, que se encargaba de algunos puntos que no fueron tomados en cuenta por las licencias BSD y GNU. En el espectro de las licencias de software libre se la puede considerar adyacente a la licencia estilo BSD, pero perfeccionada.\par
			
			\subsection{Licencia CDDL}\label{cddl}
			
				Common Development and Distribution License, también conocida como Sun Public License (SPL) versión 2, es una licencia de código abierto (OSI) y libre, producida por Sun Microsystems, basada en la Mozilla Public License o MPL, versión 1.1. La licencia CDDL fue enviada para su aprobación al Open Source Initiative el 1 de diciembre de 2004, y fue aprobada como una licencia de código abierto a mediados de enero de 2005. En el primer borrador hecho por el comité de divulgación de licencias OSI, la CDDL es una de las nueve licencias más populares, mundialmente usadas o con fuertes comunidades.\par
				
			\subsection{Licencias EPL}\label{epl}
			
				La Licencia Pública Eclipse (EPL) es una licencia utilizada por la Fundación Eclipse para su software. Sustituye a la Licencia Pública Común (CPL) y elimina ciertas condiciones relativas a los litigios sobre patentes. La Licencia Pública de Eclipse está diseñado para ser una licencia de software favorable a los negocios y cuenta con disposiciones más débiles que las licencias copyleft contemporáneas. El receptor de programas licenciados EPL pueden utilizar, modificar, copiar y distribuir el trabajo y las versiones modificadas, en algunos casos están obligados a liberar sus propios cambios.\par
			
			\subsection{Licencia Apache}\label{apsl}
			
				La licencia Apache es una licencia de software libre creada por la Apache Software Foundation (ASF). La licencia (con versiones 1.0, 1.1 y 2.0) requiere la conservación del aviso de copyright y el disclaimer, pero no es una licencia copyleft, ya que no requiere la redistribución del código fuente cuando se distribuyen versiones modificadas ni siquiera que se tengan que distribuir como software libre/open source, solo exige que se mantenga una noticia que informe a los receptores que en la distribución se ha usado código con la Licencia Apache.\par
				
			\subsection{Licencia PHP}
			
				La licencia PHP es la licencia bajo la cual se publica el lenguaje de programación PHP. De acuerdo a la Free Software Foundation es una licencia de software libre no copyleft y una licencia de código abierto según la Open Source Initiative. Debido a la restricción en el uso del término "PHP", no es compatible con la licencia GPL.
				
				Las continuas mejoras y avances dentro del lenguaje resultan de una gran comunidad de desarrolladores que contribuyen, sin obtener réditos comerciales, con:
				
				\begin{itemize}
					
					\item Código fuente.
					\item Soporte a otros usuarios a través de listas de correo.
					\item Revisión del programa en busca de errores.
					\item Notificación de fallas de seguridad y más.
				
				\end{itemize}
			
				Sobre esta base se sostiene una licencia que, justamente, asegura la libertad del lenguaje y no permite bajo concepto alguno que alguien obtenga beneficios comerciales de PHP y sea el dueño del lenguaje: éste es el espíritu de la licencia.\par
				
				Cuando se desarrolla una aplicación y se la vende a terceros el importe que se cobra no es el lenguaje de programación sino la solución a un problema, el tiempo invertido en el desarrollo, el soporte, u otro particular.\par
			
		\subsection{Licencias Creative Commons}
		
			Las licencias Creative Commons permite a los usuarios usar obras protegidas por derecho de autor sin solicitar el permiso del autor de la obra. Inicialmente, estas licencias se crearon con base en la legislación estadounidense y rápidamente fueron adaptadas a las legislaciones de los diferentes países de todo el mundo.
			
			\subsubsection{Tipos de licencias de Creative Commons}	
			
				
				Todas las licencias Creative Commons conceden ciertos derechos básicos, derecho a reproducir la obra, así como a distribuir la obra sin cargo.\par
					
				\begin{itemize}
					
					\item \textbf{Atribucion (BY)} El beneficiario de la licencia tiene el derecho de copiar, distribuir, exhibir y representar la obra y hacer obras derivadas siempre y cuando reconozca y cite la obra de la forma especificada por el autor o el licenciante.
				
					\item \textbf{No Comercial (NC)} El beneficiario de la licencia tiene el derecho de copiar, distribuir, exhibir y representar la obra y hacer obras derivadas para fines no comerciales.
					
					\item \textbf{No Derivadas (ND)} El beneficiario de la licencia solamente tiene el derecho de copiar, distribuir, exhibir y representar copias literales de la obra y no tiene el derecho de producir obras derivadas.
					
					\item \textbf{Compartir Igual (SA)} El beneficiario de la licencia tiene el derecho de distribuir obras derivadas bajo una licencia idéntica a la licencia que regula la obra original.	 
					
				\end{itemize}
				
	
				