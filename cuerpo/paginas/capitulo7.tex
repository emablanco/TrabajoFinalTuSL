	\chapter{Licencias involucradas}\label{ch:licencias}

	\section{Tipos de licencia}\vspace{0.6cm}

			\subsection{Licencia GPL}\label{gpl}
			
				Una de las licencias más utilizadas es la Licencia Pública General de GNU (GNU GPL). El autor conserva los derechos de autor y permite la redistribución y modificación del software bajo ciertos términos diseñados para garantizar que todas las versiones modificadas del software permanezcan bajo la misma licencia GNU GPL. Esto significa que no es posible crear un producto que contenga partes no licenciadas bajo GPL. La licencia GNU GPL permite la modificación y redistribución del software, pero solo bajo la condición de utilizar la misma licencia.\par
				
	
	
			\subsection{Licencias LGPL}\label{lgpl}
				
				La Licencia Pública General Reducida de GNU, más conocida por su nombre en inglés GNU Lesser General Public License (LGPL), es una licencia creada por la Free Software Foundation (FSF) que garantiza la libertad de compartir y modificar el software cubierto por ella, asegurando que el software sea libre para todos sus usuarios. Esta licencia se aplica a cualquier programa o trabajo que incluya una nota del propietario de los derechos del trabajo, estableciendo que puede ser distribuido bajo los términos de la LGPL.\par
			
			\subsection{Licencia AGPL}\label{agpl}
			
				La Licencia Pública General de Affero (AGPL) es una licencia copyleft derivada de la Licencia Pública General de GNU, diseñada específicamente para asegurar la cooperación con la comunidad en el caso de software que se ejecuta en servidores de red. Se encuentra dentro de las licencias que derivan de GNU y están destinadas a modificar los derechos de autor. La Affero GPL es en esencia una GNU GPL, pero con una cláusula adicional que impone la obligación de distribuir el software si se utiliza para ofrecer servicios a través de una red de ordenadores.

				La novedad de la AGPL es que, además de incluir las cláusulas propias de una GNU GPL, también exige la distribución del software cuando se utilice como parte del desarrollo de un nuevo software destinado a ofrecer servicios a través de una red de ordenadores. Esto significa que si alguien utiliza software AGPL como componente de un nuevo proyecto, dicho proyecto estará obligado a ser distribuido libremente.\par
				
			\subsection{Licencia Estilo BSD}\label{bsd}
			
				Se les llama así porque se utilizan en gran cantidad de software distribuido junto a los sistemas operativos BSD. Es una licencia permisiva que impone pocas condiciones sobre lo que un usuario puede hacer con el software. Bajo estas licencias, el autor mantiene la protección de copyright únicamente para renunciar a la garantía y para exigir la atribución adecuada de la autoría en trabajos derivados, pero permite la libre redistribución y modificación, incluso si esos trabajos tienen un propietario. Son muy permisivas, tanto que se pueden mezclar fácilmente con la licencia GNU GPL, con la cual son compatibles. Además, BSD permite el cobro por la distribución de objetos binarios. Sin embargo, algunas opiniones señalan que este tipo de licencia no contribuye al desarrollo de más software libre. A menudo se utiliza la siguiente analogía: "Una licencia BSD es más libre que una GPL si y solo si se opina también que un país que permite la esclavitud es más libre que otro que no la permite.\par
			
			\subsection{Licencia PSFL}\label{psfl}
			
				La Licencia de la Python Software Foundation (PSFL), anteriormente conocida como Python License, es una licencia permisiva de software libre, similar a la licencia BSD. Cumple con los requisitos de la Open Source Initiative (OSI) para ser considerada una licencia de software libre y también es compatible con la licencia GPL. A diferencia de la GPL y al igual que la mayoría de las licencias BSD, la PSFL no es una licencia copyleft, lo que significa que permite modificaciones del código fuente y la creación de trabajos derivados sin requerir que dichas modificaciones o derivados sean de código abierto. La licencia PSFL está incluida en las listas de licencias aprobadas tanto por la Free Software Foundation como por la Open Source Initiative.\par
				
			\subsection{Licencia MPL y derivadas}\label{mpl}
			
					Esta licencia es de software libre y tiene un gran valor, ya que fue el instrumento utilizado por Netscape Communications Corp. para liberar su Netscape Communicator 4.0 y comenzar el proyecto tan importante para el mundo del software libre: Mozilla. Se utiliza en una gran cantidad de productos de software libre que se utilizan a diario en diversos sistemas operativos. La Licencia Pública de Mozilla (MPL) es considerada software libre y promueve eficazmente la colaboración, evitando el efecto "viral" de la GPL. Sin embargo, la MPL no es tan permisiva como las licencias tipo BSD. Estas licencias se conocen como "copyleft débil". La NPL (posteriormente la MPL) fue la primera licencia nueva en muchos años que abordó algunos puntos que no fueron considerados por las licencias BSD y GNU. En el espectro de las licencias de software libre, se puede considerar que la MPL es adyacente a la licencia estilo BSD, pero mejorada.\par
			
			\subsection{Licencia CDDL}\label{cddl}
			
				La Licencia Común de Desarrollo y Distribución (CDDL), también conocida como Sun Public License (SPL) versión 2, es una licencia de código abierto y libre producida por Sun Microsystems. Está basada en la Licencia Pública de Mozilla (MPL), versión 1.1. La CDDL fue enviada para su aprobación al Open Source Initiative (OSI) el 1 de diciembre de 2004 y fue aprobada como una licencia de código abierto a mediados de enero de 2005. Según el primer borrador realizado por el comité de divulgación de licencias del OSI, la CDDL es una de las nueve licencias más populares, ampliamente utilizada a nivel mundial y con comunidades sólidas.\par
				
			\subsection{Licencia EPL}\label{epl}
			
				La Licencia Pública de Eclipse (EPL) es una licencia utilizada por la Fundación Eclipse para su software. Reemplaza a la Licencia Pública Común (CPL) y elimina ciertas condiciones relacionadas con litigios de patentes. La Licencia Pública de Eclipse está diseñada para ser una licencia de software favorable para los negocios y cuenta con disposiciones más flexibles que las licencias copyleft contemporáneas. Los destinatarios de programas con licencia EPL pueden utilizar, modificar, copiar y distribuir el trabajo y las versiones modificadas. En algunos casos, también pueden estar obligados a liberar sus propios cambios.\par
			
			\subsection{Licencia Apache}\label{apsl}
			
				La Licencia Apache es una licencia de software libre creada por la Apache Software Foundation (ASF). La licencia tiene varias versiones, incluyendo la 1.0, 1.1 y 2.0. La licencia requiere que se conserve el aviso de copyright y el descargo de responsabilidad, pero no es una licencia copyleft, lo que significa que no exige la redistribución del código fuente cuando se distribuyen versiones modificadas. Tampoco requiere que las versiones modificadas se distribuyan como software libre o de código abierto. Sin embargo, la licencia Apache sí exige que se mantenga una notificación que informe a los receptores que se ha utilizado código con Licencia Apache en la distribución.\par
				
			\subsection{Licencia PHP}
			
				La licencia de PHP es la licencia bajo la cual se publica el lenguaje de programación PHP. Según la Free Software Foundation, es una licencia de software libre no copyleft y, según la Open Source Initiative, es una licencia de código abierto. Sin embargo, debido a restricciones en el uso del término "PHP", no es compatible con la licencia GPL.\par 
				Las continuas mejoras y avances en el lenguaje se deben a una gran comunidad de desarrolladores que contribuyen sin obtener beneficios comerciales. Estas contribuciones incluyen:
				
				\begin{itemize}
					\item Código fuente.
    				\item Soporte a otros usuarios a través de listas de correo.
   					\item Revisión del programa en busca de errores.
    				\item Notificación de fallas de seguridad, entre otros.
				\end{itemize}
				
				Sobre esta base, se sostiene una licencia que garantiza la libertad del lenguaje y no permite que alguien obtenga beneficios comerciales exclusivos de PHP y se convierta en dueño del lenguaje. Este es el espíritu de la licencia.\par

				Cuando se desarrolla una aplicación y se vende a terceros, el monto cobrado no es por el lenguaje de programación en sí, sino por la solución a un problema, el tiempo invertido en el desarrollo, el soporte u otros aspectos específicos.\par

	
		\subsection{Licencias Creative Commons}
		
				Las licencias de Creative Commons permiten a los usuarios utilizar obras protegidas por derechos de autor sin necesidad de solicitar permiso al autor de la obra. Estas licencias se crearon inicialmente en base a la legislación estadounidense y posteriormente se adaptaron rápidamente a las legislaciones de diferentes países alrededor del mundo.
			
			\subsubsection{Tipos de licencias de Creative Commons} 
			
				Todas las licencias Creative Commons otorgan ciertos derechos básicos, como el derecho a reproducir y distribuir la obra de forma gratuita.\par 
					
						\begin{itemize}
						
    						\item \textbf{Atribución (BY):} El beneficiario de la licencia tiene el derecho de copiar, distribuir, exhibir y representar la obra, así como crear obras derivadas, siempre y cuando se reconozca y cite la obra de acuerdo a las especificaciones del autor o licenciante.

    						\item \textbf{No Comercial (NC):} El beneficiario de la licencia tiene el derecho de copiar, distribuir, exhibir y representar la obra, así como crear obras derivadas, pero únicamente para fines no comerciales.

    						\item \textbf{No Derivadas (ND):} El beneficiario de la licencia solo tiene el derecho de copiar, distribuir, exhibir y representar copias literales de la obra, sin poder crear obras derivadas.

    						\item \textbf{Compartir Igual (SA):} El beneficiario de la licencia tiene el derecho de distribuir obras derivadas bajo una licencia idéntica a la que rige la obra original.				

					\end{itemize}